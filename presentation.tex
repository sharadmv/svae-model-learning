\documentclass[10pt, compress]{beamer}

\usepackage{
    algorithm,algorithmic,amsfonts,amsmath,amssymb,bm,booktabs,color,
    enumerate,graphicx,hyperref,microtype,multicol,natbib,nicefrac,url
}
\usepackage[T1]{fontenc}
\usepackage[utf8]{inputenc}

\newcommand{\todo}[1]{\noindent\textbf{\textcolor{red}{(TODO) #1\\}}}

\newcommand{\MDP}{M}
\newcommand{\States}{\mathcal{S}}
\newcommand{\Actions}{\mathcal{A}}
\newcommand{\cost}{C}
\newcommand{\initstatedist}{\rho}
\newcommand{\discount}{\gamma}
\newcommand{\numsamp}{N}
\newcommand{\horizon}{T}
\newcommand{\dynamics}{p}
\newcommand{\policyparams}{\theta}
\newcommand{\policy}{\pi}
\newcommand{\state}{\mathbf{s}}
\renewcommand{\action}{\mathbf{a}}
\newcommand{\costsample}{c}
\newcommand{\policyobj}{\eta}
\newcommand{\dynmodel}{\hat{\dynamics}}
\newcommand{\costmodel}{\hat{\cost}}
\newcommand{\isdmodel}{\hat{\rho}}
\newcommand{\dynmat}{\mathbf{F}}
\newcommand{\dyncovar}{\Sigma}
\newcommand{\costmat}{\mathbf{C}}
\newcommand{\costvec}{\mathbf{c}}
\newcommand{\K}{\mathbf{K}}
\renewcommand{\k}{\mathbf{k}}
\newcommand{\polcovar}{\mathbf{S}}
\newcommand{\latent}{\mathbf{x}}
\newcommand{\traj}{\tau}
\newcommand{\polstepsize}{\epsilon_p}
\newcommand{\modelstepsize}{\epsilon_m}
\newcommand{\R}{\mathbb{R}}
\newcommand{\N}{\mathcal{N}}
\newcommand{\W}{\mathcal{W}}
\renewcommand{\L}{\mathcal{L}}
\newcommand{\E}{\mathbb{E}}
\newcommand{\I}{\mathbb{I}}
\newcommand{\KL}{\text{KL}}
\newcommand{\model}{\mathcal{M}}
\newcommand{\dataset}{\mathcal{D}}
\DeclareMathOperator*{\argmin}{argmin}
\DeclareMathOperator*{\argmax}{argmax}
\newcommand{\colvec}[2][.67]{%
  \scalebox{#1}{%
    \renewcommand{\arraystretch}{.67}%
    $\begin{bmatrix}#2\end{bmatrix}$%
  }
}
\newcommand{\trajectory}{\left[\state_0,\action_0,\ldots,\state_\horizon,\action_\horizon,\state_{\horizon + 1}\right]}
\newcommand{\costtrajectory}{\left[\state_0,\action_0,\costsample_0\ldots,\state_\horizon,\action_\horizon,\costsample_\horizon,\state_{\horizon + 1}\right]}
\newcommand{\latenttrajectory}{\left[\latent_0,\action_0,\ldots,\latent_\horizon,\action_\horizon,\latent_{\horizon + 1}\right]}

\usetheme{metropolis}           % Use metropolis theme

\usepackage[export]{adjustbox}
\usepackage{booktabs}
\usepackage[scale=2]{ccicons}
\usefonttheme[onlymath]{serif}

%\usemintedstyle{trac}

\title{}
\subtitle{Model-based reinforcement learning with latent representations}
\date{\today}
\author{Sharad Vikram and Marvin Zhang}
\institute{UCSD and UC Berkeley}

%\setcounter{tocdepth}{1}

\begin{document}


\maketitle

\section{Background}

\subsection{Model learning}

\begin{frame}{Model learning}
	Consider 
\end{frame}

\subsection{Reinforcement Learning}

\begin{frame}{What is reinforcement learning?}
	\only<1->{Simply put, reinforcement learning (RL) is used to solve a task from rewards or costs.}
	\begin{itemize}
		\item<2-> \textbf{Setup:} Consider an agent in an environment that acts over time ($t = 1,2,\ldots, T$)
			\begin{enumerate}
				\item<3-> Currently at state $s_t$
				\item<4-> Decide on an action $a_t = \pi(s_t)$
				\item<5-> Receive cost $c_t$
				\item<6-> Assigned next state $s_{t + 1}$
			\end{enumerate}
			\begin{center}
				\includegraphics<3>[width=0.2\textwidth]{img/agent-1.png}
				\includegraphics<4>[width=0.2\textwidth]{img/agent-2.png}
				\includegraphics<5>[width=0.2\textwidth]{img/agent-3.png}
				\includegraphics<6->[width=0.2\textwidth]{img/agent-4.png}
			\end{center}
		\item<7-> \textbf{Goal:} Learn a policy function $\pi(s)$
			that minimizes
			\begin{align*}
				\sum_{t = 1}^T c_t
			\end{align*}
	\end{itemize}
\end{frame}

\begin{frame}{Model-free reinforcement learning}
	In model-free RL, we learn a policy by directly optimizing the objective
	\begin{align*}
		\pi_\theta^*(s) = \argmin_\theta \sum_{t = 1}^T c_t.
	\end{align*}
	\pause
	\begin{center}
		\includegraphics<1->[width=0.3\textwidth]{img/model-free.png}
	\end{center}
	Examples of model-free algorithms include:
	\begin{itemize}
		\item Q-learning
		\item Trust-region policy optimization
		\item Generalized advantage estimation
	\end{itemize}
\end{frame}

\begin{frame}{Model-free vs. model-based RL}
	%Do we explicitly want to model how the environment behaves?

	An environment can be broken down into two independent components:
	\pause
	\begin{itemize}
		\item \textbf{Dynamics function}: $\state_{t + 1} \sim p(s_{t + 1}| s_t, a_t)$
			\pause
		\item \textbf{Cost function}: $c_t = C(s_t, a_t)$
	\end{itemize}
	\pause
	\metroset{block=fill}
	\begin{block}{Model-free}
		Learn a black-box policy $\pi_\theta(s_t)$ by directly optimizing $\sum_{t = 1}^T c_t$
	\end{block}
	\pause
	\metroset{block=fill}
	\begin{block}{Model-based}
		Learn a model of the environment ($\dynmodel(\state_{t+1}|\state_t,\action_t)$,  $\costmodel(\state_t,\action_t)$).
		Then, learn a policy $\pi_\theta(s_t)$ with model-predictive control (MPC).
	\end{block}
	%\begin{center}
	%\includegraphics<1->[width=0.5\textwidth]{img/model-free.png}
	%\includegraphics<2->[width=0.5\textwidth]{img/model-based.png}
	%\end{center}
\end{frame}

\begin{frame}[allowframebreaks]{References}
	\bibliography{main}
	\bibliographystyle{unsrt}
\end{frame}

\end{document}
